\documentclass[12pt, titlepage]{article}
% \linespread{1}
\title{Nonlinear Models and Their Applications}

\begin{document}
	\maketitle
\section{General Concept}
Let $F_{t-1}$ denote the information available at time $t-1$\\
Conditional mean: $\mu_{t}=E(x_{t}|F_{t-1})\equiv g(F_{t-1})$\\
Conditional variance: $\sigma_{2}^{t}=Var(x_{t}|F_{t-1}\equiv h(F_{t-1}))$\\
where $g(.)$ and $h(.)$ are well-defined functions with $h(.)>0$.

\section{TAR Model} 
A piecewise linear model in the space of a threshold
variable.
Example: 2-regime AR(1) model
$$ x_{t}=\begin{cases}
-1.5x_{t-1}+a_{t} & x_{t-1}<0 \\
0.5x_{t-1}+a_{t} & x_{t-1}\geq0 \\
\end{cases}$$
where $a_{t}$'s are iid $N(0,1)$.
Here the delay is 1 time period, $x_{t-1}$ is the threshold variable, and the threshold is 0. The threshold divides the range (or space) of $x_{t-1}$ into two regimes with Regime 1 denoting $x_{t-1}<0$.
Special features of the model: (a) asymmetry in rising and declining patterns, (more data points are positive than negative), (b) the mean of $x_{t}$ is not zero even though there is no constant term in the model, (c) the lag-1 coefficient may be greater than 1 in absolute value.   

\end{document}